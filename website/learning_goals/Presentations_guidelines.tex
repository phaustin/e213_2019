\documentclass{article}

%\usepackage[latin1]{inputenc}
\usepackage{tikz}
\usetikzlibrary{shapes,arrows}
\usepackage[margin=0.5in]{geometry}
%%%<
\usepackage{verbatim}
%\usepackage[active,tightpage]{preview}
%%\PreviewEnvironment{tikzpicture}
%\setlength\PreviewBorder{5pt}%
%%%>



\begin{document}
\pagestyle{empty}

\begin{center}
\huge{EOSC 213 - Computational methods in geological engineering}
\vspace{0.5cm}

\Large{Project presentation guidelines}
\end{center}
\vspace{1cm}

On thursday april 4$^{th}$ 2019, eachh group will have an opportunity to share and present their project. The main goals underlying these presentations are the following:

\begin{itemize}
	\item work on your scientific communication skills ;
	\item witness the work of your peers ;
	\item reflect on the progress made through this course.
\end{itemize}

While the presentations are informal, here are some guidelines to help you prepare it.

\begin{itemize}
\item Each group will perform a 3-4 minutes presentation, after which time for one or two questions/suggestions (from instructors or fellow students) will be given.
\item Presentations should include:
	\begin{enumerate}
		\item an introductory part to present the relevance of the project (context, goal, ...);
		\item a brief description of the applied methodology
		\item a highlight of the obtained results, including a critical review of their relevance.
	\end{enumerate}
\item Quality of the presentations will contribute to the overall project result. Main assessment criteria:
	\begin{enumerate}
		\item respect of the time limitations
		\item clarity of the presentation and communication
	\end{enumerate}
\end{itemize}




\end{document}